\chapter{Observace, měření a reprezentace}
K pozorování meteorů a jejich záznamu se v současnosti používají tři přístupy: fotografické snímky, videozáznamy a radarová měření.

Ve všech třech přístupech je cílem sledovat a zaznamenávat celou oblohu nebo alespoň její velkou část. Například u fotografického přístupu se používá rybí oko \cite{ceplecha} -- čočka nebo objektiv, který je schopný zobrazit celou oblohu na jeden snímek. Cenou za takto širokoúhlý snímek je velké zkreslení obrazu, to lze ale pro účely měření matematicky odstranit \cite{ceplecha}.

\todo{Radarová měření dle \cite{radiosurvey}}

Fotografické snímky a videozáznamy jsou z hlediska měření velmi blízké: Videozáznam je v principu pouze série fotografií, v minulém století se ale využívalo spíše opačného přístupu, kdy se několik fotografických záběrů zaznamenalo na jeden snímek \cite{ceplecha}. Oba přístupy tedy dávají průběh polohy (a případně i luminosity) meteoru v čase. Konkrétně pro metody identifikace meteorických rojů potřebujeme právě dráhu (polohu) a rychlost meteoru \cite{ceplecha}, abychom zjistili orbitální dráhu meteoroidu. Důkladněji se zpracování fotografických snímků budeme věnovat v sekci \ref{sec:foto}.

\section{Elementy dráhy}
\note{Rozebrat geometrii orbitálních drah a přehledně rozepsat a rozkreslit význam elementů dráhy.}
Orbitální dráhy jsou v prvním přiblížení elipsy, které jsou nakloněné v prostoru. Jedním z ohniskových bodů je vždy těžiště (Sluneční) soustavy, pro určení dráhy tedy stačí 5 parametrů \cite{newapproach}.

Dva z parametrů popisují tvar elispy; její velikost a excentricitu \cite{newapproach}. Jako excentricitu používáme lineární excentricitu $e$. Excentricita náleží do intervalu $e\in\left[0,1\right]$,\footnote{Excentricita může být také $>1$ pro hyperbolické dráhy. \ask{Hyperbolické, předpokládám, ignorujeme, protože by patřily do sporadického pozadí automaticky (nemají mateřský objekt ve Sluneční soustavě).}} a udává, jak blízká kružnici tato dráha je (viz obrázek \ref{img:excentricity}).

\todo{Ilustrace excentricit}

Druhý parametr udává velikost, poloměr, elipsy. Zde používáme buďto délku hlavní poloosy $a$ nebo vzdálenost perihelionu $q$ (efektivně vzdálenost okraje elipsy od ohniska). Jejich vztah ilustruje obrázek \ref{img:elipsa} a mezi oběma lze převádět pomocí rovnice \cite{ceplecha}
$$
    q=a(1-e)\text{.}
$$

\todo{Obrázek vztahu mezi $a$ a $q$}

\section{Fotografická měření\label{sec:foto}}
\note{}