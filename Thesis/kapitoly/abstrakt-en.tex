{\small
From photographic, video, or radar observations of meteors we are able to determine the orbit of a meteorite before it's encounter with Earth by analysing the trajectory of the meteor. We then decide which meteor shower the observed meteor belongs to based on the orbital elements of this orbit.\\
In the second half of the last century, several methods, collectively known as $D$-criteria, have been devised for this exact purpose. These have stemmed from the method presented by Southworth and Hawkins, which calculates an orbital dissimilarity measure, a numerical value representing how different orbits of two meteoroids are, and compares it with some fixed cutoff value. If the measure is less than the cutoff, the meteors belong to the same meteor shower.\\
In this thesis we describe the $D_\text{SH}$, $D_\text{D}$, $D_\text{H}$ and $D_\text{N}$ criteria and discuss their properties. As the practical part, we have created a software tool written in Python which is used to apply these criteria to real-world data. We also explain the techniques of capturing and analysis of photographic and video records of meteors in great detail.
}