\chapwithtoc{Úvod}% MARK: Úvod

\textit{Meteory} jsou světelné stopy \note{jevy} vznikající při průletu menších těles -- \textit{meteoroidů} -- atmosférou Země či jakéhokoliv jiného\footnote{Ačkoli je definice \cite{meteorastro} velmi obecná v tělese, kam meteoroid padá, budeme se zajímat pouze o meteory pozorované v atmosféře Země.} vesmírného tělesa \cite{meteorastro}.\footnote{\textit{Meteorit} je posledním z matoucích termínů v meteorické astronomii, jedná se o pevný zbytek meteoroidu po průletu atmosférou, který dopadne na povrch Země nebo jiného vesmírného tělesa \cite{meteorastro}. Ne všechny meteoroidy zanechávají meteority, některé se mohou v atmosféře úplně odpařit.} Samotné meteory jsou velkolepým úkazem, z astronomického hlediska je však více než jednotlivé meteory zajímavá souvislost mezi nimi: Jelikož meteoroidy vznikají často jako úlomky z komet či planetek \cite{meteorastro}\cite{cometassoc}, ukazuje se, že lze pozorované meteory často zařadit do "`rodin"' zvaných \textit{meteorické roje}.

Meteory můžeme buďto přiřadit do nějakého roje, nebo do tzv. \textit{sporadického pozadí}. \ask{Následuje spekulace. Je nějaké dobré místo, kde o historii zjistit více?} Historicky se meteory pozorovaly hlavně okem, případně s pomocí dalekohledů, a do rojů se zařazovaly podle ročního období (nebo konkrétněji, měsíce) a směru, odkud se jevily přilétat. Příchod a rozvoj obrazových záznamových médií od prvních fotografických emulsí až po současné videozáznamy však umožnil meteory zaznamenávat a detailně studovat jejich vlastnosti. Se zvyšováním dostupnosti fotografických zařízení, vývojem citlivějších záznamových médií a nástupem automatizace významně narostl počet pozorovacích stanic po celé Zemi a tím pádem ohromný nárůst spatřených a zaznamenaných meteorů.

Tento velký objem pozorovaných meteorů společně s potřebou zpřesňování měření a klasifikace pro další vývoj astronomie vedl ke snaze provádět zařazování jednotlivých meteorů do meteorických rojů (či do sporadického pozadí) rigorózními způsoby založenými hlavně na dráze meteoroidu \cite{dsh} v meziplanetárním prostoru -- jelikož jsou meteoroidy často úlomky z jistého mateřského tělesa \cite{meteorastro}\cite{cometassoc}, dá se předpokládat, že úlomky ze stejného tělesa budou mít podobné dráhy \cite{dsh}.

Většina používaných metod zařazování meteorů staví na metodě Southwortha a Hawkinse \cite{dsh}, která je založená na výpočtu míry "`orbitální odlišnosti"' z elementů dráhy meteoroidů. Těmito metodami se budeme dále zabývat, nicméně je vhodné zmínit, že s nástupem informačních technologií a strojového učení ("`machine-learning"') se nepochybně objevily i metody založené právě na strojovém učení. Jelikož ale strojové učení funguje v principu do velké míry jako "`černá skříňka,"' není prakticky příliš možné popisovat vlastnosti či fyzikální pozadí těchto metod.

\medskip

\note{Práce se skládá z částí ... a poslední kapitola je praktická část -- představení programu...}