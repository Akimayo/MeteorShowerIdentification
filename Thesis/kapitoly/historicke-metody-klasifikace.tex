\chapter{Historické metody klasifikace}% MARK: Historické metody klasifikace
Nejjednodušší a také historicky používanou metodou klasifikace meteorických rojů bylo pouhé porovnání polohy radiantu (bodu na nebeské sféře, odkud meteory jednoho roje zjevně přilétají \cite{glossary}) a geocentrické rychlosti \cite{radiosurvey}. Z této metody také vychází způsob pojmenování meteorických rojů: Roje dostávají jméno podle hvězdy, která je na nebi nejblíže radiantu \cite{dsh}. Získáváme tak názvy jako např. \textit{Perseidy}, \textit{Jižní $\delta$ Aquaridy} nebo \textit{Severní Tauridy}.

Bylo ale rozhodnuto, že pro prvotní klasifikaci meteorických rojů budou používány elementy dráhy, které lépe vystihují původ meteoroidů a nemění se v čase tak rychle jako polohy radiantů \cite{radiosurvey}.

\medskip

Všechna následující kritéria pracují na jednoduchém principu: Vezmeme elementy dráhy dvou meteoroidů a pomocí daného kritéria ověříme, zda patří do stejného roje či nikoliv. Máme-li meteoroidů více, po dvojicích ověřujeme kritérium a sestavujeme tak třídy ekvivalence meteoroidů -- kandidáty na meteorické roje. Obdobně, máme-li je k dispozici, můžeme využít střední elementy dráhy známých meteorických rojů a nově pozorovaný meteor zkusit přiřadit do některého z nich.

Pokud třída ekvivalence má dostatek prvků (více než 2 meteoroidy \cite{radiosurvey}), jedná se pravděpodobně o meteorický roj. Podle četnosti se meteorické roje dále děli na hlavní a vedlejší. Pokud však meteoroid tvoří třídu ekvivalence jen sám nebo jen s několika málo dalšími meteoroidy, zařazujeme jej do \textit{sporadického pozadí}, tedy meteoroidů, které se oddělily samostatně, jsou již staré a byly příliš perturbovány pohybem planet \cite{dsh}, nebo nepocházejí ze Sluneční soustavy.

\section{Přímé porovnávání elementů dráhy}% MARK: Přímé porovnání elementů dráhy
\citeauthor{radiosurvey} nebyl prvním, kdo přišel s úspěšným kritériem přiřazování meteorů do meteorických rojů, ale v \cite{radiosurvey} publikoval jednoduše uchopitelné a efektivní kritérium: Zda se elementy dráhy dvou meteoroidů liší o méně než jisté stanovené hodnoty.

Kritérium zavedené \citeauthor{radiosurvey}em bere pro dva meteoroidy elementy dráhy $a_i$, $e_i$, $i_i$ a $\nu_i$ a říká, že tyto dva meteoroidy spadají do stejného roje, pakliže splňují všechny následující podmínky \cite{radiosurvey}:
\begin{eqnarray}
    \left|\frac{1}{a_1}-\frac{1}{a_2}\right| &\le& 0{,}15 \text{,}\\
    \left|e_1-e_2\right| &\le& 0{,}07 \text{,}\\
    \left|i_1-i_2\right| &\le& 7^\circ \text{,}\\
    \left|\nu_1-\nu_2\right| &\le& 7^\circ \label{eqn:history:anomaly}\text{.}
\end{eqnarray}
Poslední podmínka je ještě doplněna o případ, kdy je excentricita alespoň jednoho meteoroidu malá: $e<0{,}3$ \cite{radiosurvey}. Mění se pak na \cite{radiosurvey}
\begin{equation}
    \tag{\ref{eqn:history:anomaly}*}
    \left|\nu_1-\nu_2\right| \le 7^\circ+100(0{,}3^\circ-e)\text{.}
\end{equation}
\citeauthor{radiosurvey} ale dodává, že v praxi se toto upřesnění neuplatní, jelikož nebyly známy žádné meteorické roje s $e<0{,}4$ \cite{radiosurvey}. Dále by se žádné dva meteoroidy z daného meteorického roje neměly lišit o více než dvojnásobek limitu \cite{radiosurvey}.

\section{Míra orbitální odlišnosti, $D$-kritérium}% MARK: Míra orbitální odlišnosti, D-kritérium
% \cite{dsh} \cite[220]{radiosurvey} \cite[604]{remarks} \cite{newapproach} \cite[623]{galligan}
\citeauthor{dsh} navrhli koncept číselné míry orbitální odlišnosti dvou meteoroidů: Zavedli funkci $D(A,B)$, jejíž hodnota je mírou odlišnosti orbitů meteoroidů $A$ a $B$, a obdobně funkci $D(M,N)$, která má stejný matematický předpis, ale porovnává dráhu meteoroidu $N$ se střední dráhou známého roje $M$ \cite{dsh}. S jejich pomocí předložili tři možné metody klasifikace:
\begin{enumerate}
    \item Funkci $D(A,B)$ aplikujeme na všechny dvojice meteoroidů a ověřujeme, zda je její hodnota menší než jistá hraniční hodnota $D_S$ \cite{dsh}. Tvoříme tak třídy ekvivalence (podobně jako v \citeauthor{radiosurvey}ově metodě) a roj identifikujeme jako třídu ekvivalence s dostatečně velkým počtem prvků.
    \item Samostatný meteoroid $N$ dosazujeme do funkce $D(M,N)$ se všemi známými roji $M$ a příslušnost do daného meteorického roje ověřujeme podmínkou, zda je číselná hodnota menší než jistá hraniční hodnota $D_M$. Alternativně, meteorický roj definujeme jako všechny meteoroidy $N$ takové, že hodnota $D(M,N)$ je menší než hraniční hodnota $D_M$ \cite{dsh}.
    \item Definujeme vícedimenzionální metrický prostor, ve kterém je každý meteoroid definovaný jedním bodem na základě jeho elementů dráhy a vzdálenost mezi body je dána funkcí $D(A,B)$ \cite{dsh}. Meteorické roje pak můžeme identifikovat klastrovacími algoritmy (hledání oblastí metrického prostoru s vysokou hustotou bodů).\footnote{V době publikace \cite{dsh} (rok \citeyear{dsh}) nebyl k dispozici dostatečný počet pozorování, aby metody klastrování byly efektivní \cite{dsh}. Jedná se ale o poutavou myšlenku a jednoduše představitelný mechanismus klasifikace.}
\end{enumerate}

V praxi se setkáváme primárně s druhou metodou, jelikož jsou v současnosti dráhy meteorických rojů již dobře proměřeny a u nových observací se primárně snažíme přiřadit meteor do existujícího roje (či vyloučit a zařadit do sporadického pozadí). První metoda je ovšem stále žádoucí, jelikož umožňuje identifikovat nové meteorické roje.

\subsection{Obecná formulace míry orbitální odlišnosti}% MARK: Obecná formulace míry orbitální odlišnosti
Funkce $D(A,B)$ je zavedena velmi obecně jako \cite{dsh}
\begin{equation}
    D(A,B)=\sqrt{
    \sum_{j=1}^{k}{c_j^2\left[ \varrho_j(A)-\varrho_j(B) \right]^2}
    }\text{.}
\end{equation}
Jedná se o sumu rozdílů jistých číselných vlastností orbitů $\varrho_j$ vážených funkcemi $c_j$. \citeauthor{dsh} pracují s $k=5$ vlastnostmi, aby zahrnuli pět stabilních elementů dráhy (všechny kromě pravé anomálie $\nu$), tento přístup je ale sporný, jelikož (minimálně u meteoroidů) tyto elementy dráhy nejsou všechny nezávislé -- konkrétně existuje závislost mezi $\Omega$ a $\omega$ \cite{remarks}.

Počet těchto vlastností je ale volitelný a dokonce samotné $\varrho_j$ a $c_j$ \citeauthor{dsh} stanovují pouze jako jimi použitý příklad a doporučují je vylepšovat. Vlastnostmi $\varrho_j$ mohou býti samotné elementy dráhy $q/a,e,i,\Omega,\omega$ nebo jejich funkce a obdobně $c_j$ mohou býti konstantami či také funkcemi elementů dráhy. Váhy $c_j$ by však měly být nepřímo úměrné očekávané standardní odchylce příslušné $\varrho_j$ \cite{dsh}.

Přirozeně, od volby jednotlivých hodnot se budou odvíjet také hranice $D_S$ a $D_M$.

\smallskip

Vysoká flexibilita této míry orbitální odlišnosti vedla k jejímu rozšíření a většina metod klasifikace popisovaných dále v této práci z ní vychází, liší se pouze v některých $\varrho_j$ a $c_j$ a v hranicích.

\subsection{Kritérium $D_{SH}$}% MARK: Kritérium DSH
Míra odlišnosti, kterou pro svá měření používali \citeauthor{dsh}, používá excentricity $e$, vzdálenosti perihelia $q$\footnote{Narozdíl od délky velké poloosy je vzdálenost perihelia přímo měřitelná a nabývá menších hodnot \cite{dsh}, tedy má menší nejistoty a nepřeváží ostatní elementy dráhy.} a délky tětiv mezi rovinami orbitů a mezi argumenty perihelia. Délka tětivy je pro malé úhly velmi blízká délce oblouku (úhlu v radiánech) a na intervalu $\left[0,2\pi\right)$ nabývá pouze kladných hodnot \cite{dsh}.

Tato míra má tvar \cite{dsh}\cite{remarks}
\begin{equation}
    \begin{aligned}
        D_{SH}^2(A,B)=\; & (q_B-q_A)^2                                                                         \\
                         & +(e_B-e_A)^2                                                                        \\
                         & +\left( 2\sin{\frac{I_{AB}}{2}} \right)^2                                           \\
                         & +\left( \frac{e_B-e_A}{2} \right)^2\left( 2\sin{\frac{\Pi_{AB}}{2}} \right)^2 \text{,}
    \end{aligned}
\end{equation}
kde $I_{AB}$ je úhel mezi orbitálními rovinami definovanými inklinací a délkou vzestupného uzlu a $\Pi_{AB}$ je rozdíl argumentů perihelia měřených od průsečnice rovin. Tětiva úhlu mezi rovinami se vypočte \cite{dsh}
\begin{equation}
    \left( 2\sin{\frac{I_{AB}}{2}} \right)^2=\left( 2\sin{\frac{i_B-i_A}{2}} \right)^2
    + \sin{i_A}\sin{i_B}\left( 2\sin{\frac{\Omega_B-\Omega_A}{2}} \right)^2
\end{equation}
a rozdíl argumentů perihelia \cite{dsh}
\begin{equation}
    \Pi_{AB}=\omega_B-\omega_A+2\arcsin{\left( \frac{\cos{\frac{i_A+i_B}{2}}\sin{\frac{\Omega_B-\Omega_A}{2}}}{\cos{\frac{I_{AB}}{2}}} \right)} \text{.}
    \label{eqn:history:pi_ba}
\end{equation}
Jsou-li inklinace $i_A$ a $i_B$ malé, můžeme poslední výraz zjednodušit na \cite{dsh}
\begin{equation}
    \tag{\ref{eqn:history:pi_ba}*}
    \Pi_{AB}=(\Omega_B+\omega_B)-(\Omega_A+\omega_A) \text{.}
\end{equation}

Použitím této míry na členy známých meteorických rojů \citeauthor{dsh} ukázali, že vhodnou hraniční hodnotu je pro střední dráhu i pro dvojice meteorů $D_S=D_M=0{,}2$ \cite{dsh}. Je ale doporučeno používat spíše hodnotu $D_S=0{,}09$ pro $i<10^\circ$ a $D_S=0{,}12$ pro $10^\circ\le i<90^\circ$ \cite{galligan}.