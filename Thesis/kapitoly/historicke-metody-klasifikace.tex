\chapter{Historické metody klasifikace}% MARK: Historické metody klasifikace
Nejjednodušší a také historicky používanou metodou klasifikace meteorických rojů bylo pouhé porovnání polohy radiantu (bodu na nebeské sféře, odkud meteory jednoho roje zjevně přilétají \cite{glossary}) a geocentrické rychlosti \cite{radiosurvey}. Bylo ale rozhodnuto, že pro úvodní klasifikaci meteorických rojů budou používány elementy dráhy, které lépe vystihují původ meteoroidů a nemění se v čase tak rychle jako polohy radiantů \cite{radiosurvey}.

\ask{Podle polohy radiantu se meteorické roje pojmenovávají. Nenašel jsem ale na to žádný dostatečně oficiální zdroj. Mám to uvádět?}

\medskip

Všechna následující kritéria pracují na jednoduchém principu: Vezmeme elementy dráhy dvou meteoroidů a pomocí daného kritéria ověříme, zda patří do stejného roje či nikoliv. Máme-li meteoroidů více, po dvojicích ověřujeme kritérium a sestavujeme tak třídy ekvivalence meteoroidů -- kandidáty na meteorické roje. Obdobně, máme-li je k dispozici, můžeme využít střední elementy dráhy známých meteorických rojů a nově pozorovaný meteor zkusit přiřadit do některého z nich.

Pokud třída ekvivalence má dostatek prvků (více než 2 meteoroidy \cite{radiosurvey}), jedná se pravděpodobně o meteorický roj. Podle četnosti se meteorické roje dále děli na hlavní a vedlejší. Pokud však meteoroid tvoří třídu ekvivalence jen sám nebo jen s několika málo dalšími meteoroidy, zařazujeme jej do \textit{sporadického pozadí}, tedy meteoroidů, které se oddělily samostatně, jsou již staré a byly příliš perturbovány pohybem planet \cite{dsh}, nebo nepocházejí ze Sluneční soustavy.

\section{Přímé porovnávání elementů dráhy}% MARK: Přímé porovnání elementů dráhy
\citeauthor{radiosurvey} nebyl prvním, kdo přišel s úspěšným kritériem přiřazování meteorů do meteorických rojů, ale v \cite{radiosurvey} publikoval jednoduše uchopitelné a efektivní kritérium: Zda se elementy dráhy dvou meteoroidů liší o méně než jisté stanovené hodnoty.

Kritérium zavedené \citeauthor{radiosurvey}em bere pro dva meteoroidy elementy dráhy $a_i$, $e_i$, $i_i$ a $\nu_i$ a říká, že tyto dva meteoroidy spadají do stejného roje, pakliže splňují všechny následující podmínky \cite{radiosurvey}:
\begin{eqnarray}
    \left|\frac{1}{a_1}-\frac{1}{a_2}\right| \le& 0{,}15 \text{,}\\
    \left|e_1-e_2\right| \le& 0{,}07 \text{,}\\
    \left|i_1-i_2\right| \le& 7^\circ \text{,}\\
    \left|\nu_1-\nu_2\right| \le& 7^\circ \label{eqn:history:anomaly}\text{.}
\end{eqnarray}
Poslední podmínka je ještě doplněna o případ, kdy je excentricita alespoň jednoho meteoroidu malá: $e<0{,}3$ \cite{radiosurvey}. Mění se pak na \cite{radiosurvey}
\begin{equation}
    \tag{\ref{eqn:history:anomaly}*}
    \left|\nu_1-\nu_2\right| \le 7^\circ+100(0{,}3^\circ-e)\text{.}
\end{equation}
\citeauthor{radiosurvey} ale dodává, že v praxi se toto upřesnění neuplatní, jelikož nebyly známy žádné meteorické roje s $e<0{,}4$ \cite{radiosurvey}. Dále by se žádné dva meteoroidy z daného meteorického roje neměly lišit o více než dvojnásobek limitu \cite{radiosurvey}.

\section{Funkce orbitální odlišnosti, $D$-kritérium}% MARK: Funkce orbitální odlišnosti, D-kritérium
% \cite{dsh} \cite[220]{radiosurvey} \cite[604]{remarks} \cite{newapproach} \cite[623]{galligan}
\note{\cite{dsh}. Obecná míra odlišnosti. Míra odlišnosti elementů dráhy. Odlišnost dvou orbitů a odlišnost od průměrného orbitu.}