\chapter{Programový nástroj pro identifikaci meteorických rojů\label{cpt:practical}}% MARK: Praktická část
Praktická část této práce spočívá ve vytvoření počítačového programu, který výše popisovaná kritéria aplikuje na reálná data a umožňuje tím jak určovat příslušnost nově pozorovaných meteorů do meteorických rojů, tak hledat meteorické roje nové.

Program byl vytvořen v jazyce \textit{Python}, který je vysoce přenositelný mezi operačnímy systémy a ve fyzice často používaný, cílové skupině uživatelů by tedy měl být srozumitelný. Úplný zdrojový kód programu je dostupný v elektronických přílohách této práce a také veřejně v repozitáři \\\href{https://github.com/Akimayo/MeteorShowerIdentification}{github.com/Akimayo/MeteorShowerIdentification}, kam budou v případě nalezení chyb nahrávány opravené verze.

Součástí zdrojového kódu je také informační soubor \texttt{README.md}, který obsahuje základní instrukce pro použití programu, a kód samotný je bohatě komentovaný. Tato kapitola dokumentuje způsoby použití programu, vysvětluje jeho vstupy a výstupy a popisuje použité techniky zpracování dat.

\section{Základní použití}% MARK: Základní použití
Program má tři režimy běhu:
\begin{itemize}
    \item Porovnat jeden meteor se souborem meteorických rojů,
    \item porovnat soubor meteorů se souborem meteorických rojů a
    \item vzájemně porovnat meteory v souboru a nalézt nové meteorické roje.
\end{itemize}

Meteory i meteorické roje se zadávají pomocí elementů dráhy a případně názvu. V případě porovnání jednoho meteoru se elementy dráhy zadávají přímo do argumentů spuštění programu, v ostatních případech jsou elementy dráhy čteny ze vstupního souboru ("`srovnávaného souboru"'). Meteorické roje jsou vždy čteny ze souboru ("`referenčního souboru"'), zde je ale možno využít vestavěné databáze elementů dráhy známých meteorických rojů -- to se provádí zadáním \texttt{default} namísto cesty ke vstupnímu souboru s meteorickými roji.

\medskip

Následující stránka ukazuje nápovědu vestavěnou v programu, která je vypsána při spuštění programu příkazem \texttt{meteors ---help}. Ukazuje všechny argumenty, které program přijímá. Konkrétní příklady příkazů pro spuštění programu jsou dále popsány v sekci \ref{sec:practical:examples}.

\pagebreak
\begin{Verbatim}[commandchars=\\\{\},gobble=4,formatcom=\footnotesize,codes={\catcode`$=3\catcode`^=7},frame=lines]
    \textbf{METEOR SHOWER IDENTIFICATION CONSOLE}
    meteors [<compared_file> [<reference_file>|"default"]] [...options]

     \textbf{Configuration}
    ===============
    --config, -c
              Creates a {\color{gray}.meteorrc} configuration file
            for the current directory.
              This file is in the YAML format.
    -0
              Disregard {\color{gray}.meteorrc} configuration files.

     \textbf{Single Orbit}
    ==============
    -e {\color{gray}<eccentricity>}
    -q {\color{gray}<perihel_dist>}
    -i {\color{gray}<inclination>}
    -w {\color{gray}<arg_perihel>}
    -O {\color{gray}<asc_lat>}
              Set the orbital elements of a meteor orbit
            with perihelion distance \textbf{q}.
    -e {\color{gray}<eccentricity>}
    -a {\color{gray}<smajor_axis>}
    -i {\color{gray}<inclination>}
    -w {\color{gray}<arg_perihel>}
    -O {\color{gray}<asc_lat>}
              Set the orbital elements of a meteor orbit
            with semimajor axis length \textbf{a}.

     \textbf{Input/Output}
    ==============
    --criteria {\color{gray}<criteria>}, -r {\color{gray}<criteria>}
              Set the D-criteria to be used.
              Takes a comma-separated list. Supported are
            the follwing values: {\color{gray}"sh"}, {\color{gray}"d"}, {\color{gray}"h"}, {\color{gray}"n"}
    --output {\color{gray}<path>}, -o {\color{gray}<path>}
              Directs the results to a file. Omitting
            this option will write results to the console.

     \textbf{Miscellaneous}
    ===============
    --force, -f
              Forces a potentially unsafe operation.
    --verbose
              Enables verbose logging to console.
    --help, -h
              Prints this help text.
\end{Verbatim}
\pagebreak

Režim běhu programu je určen zadanými vstupy. Jsou-li zadány elementy dráhy meteoru a referenční soubor (či \texttt{default} pro vestavěnou databázi meteorických rojů), je tento jeden meteor porovnán se všemi meteorickými roji v referenčním souboru. Jsou-li zadány dva vstupní soubory, první z nich je považován za srovnávaný a druhý za referenční soubor a každý meteor ve srovnávaném souboru je porovnán se všemi meteorickými roji z referenčního souboru. Je-li zadán pouze jeden vstupní soubor, je provedeno porovnání všech dvojic meteorů ve vstupním souboru a výsledky porovnávání jsou použity k identifikaci meteorických rojů. Porovnáváním zde myslíme, že je na dvojici meteor--meteorický roj či meteor--meteor aplikováno $D$-kritérium a vyhodnoceno jeho splnění.

\medskip

Spuštěním programu s argumentem \texttt{---config} vygeneruje v aktuálním adresáři soubor \texttt{.meteorrc}, který slouží jako konfigurační soubor programu při spuštění z tohoto adresáře. Tento konfigurační soubor může obsahovat všechny možnosti pro běh programu, které by se jinak zadávaly pomocí argumentů, a také několik dalších možností, jako je určení vlastních hraničních hodnot pro $D$-kritéria či nastavení formátu vstupních souborů

Konfigurační soubor \texttt{.meteorrc} je ve formátu YAML a bylo pro něj vytvořeno schéma, které popisuje strukturu tohoto souboru. Tato kombinace umožňuje efektivní úpravu konfiguračního souboru pomocí chytrých editorů, které uživateli na základě schématu nabídnou dostupné možnosti a k nim i slovní vysvětlení, které je ve schématu obsaženo. Schématem se navíc ověřuje správnost konfiguračního souboru, a to jak přímo v editoru, tak při spuštění programu.

\subsection{Formát vstupních souborů}% MARK: Formát vstupních souborů
Vstupními soubory jsou textové soubory se sloupci pevné šířky a jedním meteorem či meteorickým rojem na řádek. Mezi každými dvěma sloupci musejí být alespoň dvě mezery. Volitelně mohou mít souboru řádek s hlavičkou, která dává jednotlivým sloupcům název, jinak je pro identifikaci sloupců použito jejich pořadové číslo (počítáno od nuly). Vstupní soubor může vypadat například takto:
\begin{Verbatim}[commandchars=\\\{\},gobble=4,formatcom=\footnotesize,codes={\catcode`$=3\catcode`^=7},frame=lines,reflabel=code:practical:input,label=DAT]
    CODE         ECC         PERIH       INCL        ARGUP        NODE
    10C12041     0.90966     0.12194     21.9471     327.0698     260.5980
    10C12049     0.90928     0.12834     22.7444     325.6990     260.6069
    16707001     0.67934     0.96083     43.9355     149.7783     105.9640
\end{Verbatim}

Takto strukturovaný soubor program očekává, resp. tyto sloupce vyžaduje, nebylo-li specifikováno jinak v konfiguračním souboru. Výchozí pojmenování sloupců, jejich význam, dodatečné sloupce a klíče konfiguračního souboru, pomocí kterých lze použít sloupce s jinými názvy, jsou uvedeny v tabulce \ref{tbl:practical:columns}.

Úhly se zadávají ve stupních a vzdálenosti v astronomických jednotkách.

\begin{table}[ht]
    \centering
    \caption[Možné sloupce vstupních souborů a jejich výchozí názvy]{Možné sloupce vstupních souborů, jejich výchozí názvy a klíče konfiguračního souboru}
    \begin{tabular}{|ccc|}
        \hline
        \textbf{Význam}   & \textbf{Výchozí název} & \textbf{Klíč v konfiguraci} \\
        \hline
        název$^*$         & --                     & \texttt{name}               \\
        kód, označení$^*$ & \texttt{CODE}          & \texttt{code}               \\
        $q^{**}$          & \texttt{PERIH}         & \texttt{perihelionDistance} \\
        $a^{**}$          & --                     & \texttt{axisLength}         \\
        $e$               & \texttt{ECC}           & \texttt{eccentricity}       \\
        $i$               & \texttt{INCL}          & \texttt{inclination}        \\
        $\omega$          & \texttt{ARGUP}         & \texttt{perihelionArgument} \\
        $\Omega$          & \texttt{NODE}          & \texttt{ascNodeLongitude}   \\
        \hline
    \end{tabular}
    {\footnotesize\\
    *) volitelné hodnoty pro jednodušší identifikaci meteorů či rojů\\
    **) tyto možnosti jsou alternativní, lze použít pouze jednu z těchto možností
    }
    \label{tbl:practical:columns}
\end{table}

Příklad i výchozí názvy sloupců jsou převzaty z dat, která nám byla poskytnuta za účelem vývoje programu. Má-li uživatel vstupní data s jinak pojmenovanými sloupci, může buďto sloupce ve svém vstupním souboru ručně přejmenovat, nebo může v konfiguračním souboru nastavit jiné názvy (nebo pořadová čísla) sloupců například takto:
\begin{Verbatim}[commandchars=\\\{\},gobble=4,formatcom=\footnotesize,frame=lines,label=YAML]
    {\color{blue}inputs}:
      {\color{blue}compared}: {\color{teal}# Předefinování sloupců srovnávaného souboru}
        {\color{blue}path}: {\color{brown}./orbits.dat} {\color{teal}# Cesta ke srovnávanému souboru}
        {\color{blue}columns}:
          {\color{blue}name}: {\color{brown}NAME}
          {\color{blue}axisLength}: {\color{brown}AXIS}
          {\color{blue}eccentricity}: {\color{brown}ECCENTR}
          {\color{blue}inclination}: {\color{brown}INCLIN}
          {\color{blue}perihelionArgument}: {\color{brown}ARG_PERI}
          {\color{blue}ascNodeLongitude}: {\color{brown}ASC_NODE}
\end{Verbatim}

\subsection{Formát výstupů}% MARK: Formát výstupů
Jako příklad použijeme vstupní soubor ze strany \pageref{code:practical:input} a program spustíme v režimu vzájemného porovnání meteorů a nalezení rojů. V tomto režimu dostaneme méně výsledků porovnání meteorů, pro ilustraci formátu výstupu je toto ale kompaktní a postačující příklad. Výsledkem takového běhu bude výstup\footnote{Řádek výstupu byl zkrácen pro lepší čitelnost.}
\begin{Verbatim}[commandchars=\\\{\},gobble=4,formatcom=\footnotesize,frame=lines,reflabel=code:practical:output]
    ***** Result 10C12041 *****
    \eCheck  10C12049   D(sh)=0.026 (cutoff 0.09)  D(d)=0.027 (cutoff 0.09) [...]
    \eCross  16707001   D(n)=0.562 (cutoff 0.08)
    \eCross  16707001   D(d)=0.889 (cutoff 0.09)
    \eCross  16707001   D(h)=1.391 (cutoff 0.1)
    ***** Result 10C12049 *****
    \eCross  16707001   D(n)=0.559 (cutoff 0.08)
    \eCross  16707001   D(d)=0.883 (cutoff 0.09)
    \eCross  16707001   D(h)=1.399 (cutoff 0.1)
    ***** Result 16707001 *****
    
    ***** Shower 10C12041 *****
    10C12041
    10C12049    
\end{Verbatim}

\medskip

První část výstupu obsahuje výsledky porovnání jednotlivých meteorů a je společná pro všechny režimy běhu programu. Pro každý meteor se v souboru nachází sekce nadepsaná klíčovým slovem \texttt{Result} a kódem meteoru, která na následujících řádcích obsahuje výsledky porovnání tohoto meteoru s meteorickým rojem či jiným meteorem. Jednotlivé informace na každém řádku jsou rozděleny do sloupců oddělených tabulátory, což dovoluje jednoduchou automatizaci případného dalšího zpracování.

V prvním sloupci se nachází symbol \eCheck\hspace{4pt}(Unicode \texttt{U+2714 U+FE0F} \cite{unicode}) či \eCross\hspace{4pt}(Unicode \texttt{U+2716 U+FE0F} \cite{unicode}) symbolizující, zda bylo kritérium pro tento meteor či roj splněno. Využití symbolů Emoji je v programování nekonvenčí, ba i nedoporučené, zde ale tyto symboly slouží pro rychlé vizuální určení, zda bylo kritérium splněno. Z hlediska dalšího zpracování se jedná pouze o textové řetězce obsahující dva znaky, kde navíc splnění kritéria lze zjistit ze znaku prvního.

Druhý sloupec obsahuje kód meteoru či meteorického roje, se kterým byl meteor porovnáván.

Zbylé sloupce obsahují hodnoty jednotlivých $D$-funkcí a použitou hraniční hodnotu (označeno \texttt{cutoff}). Pro meteory či roje splňující alespoň jedno kritérium jsou vypsána všechna splněná kritéria.

\smallskip

Výstup obsahuje až tři meteory či meteorické roje, které kritérium nesplňují. Jedná se vždy o neúspěšná porovnání, která však byla nejblíže splnění kritéria, a umožňují případné ruční přiřazení hraničních případů do rojů či úpravu hraničních hodnot kritérií.

V případě režimu hledání meteorických rojů se daná dvojice meteorů porovnává pouze jednou a výsledek porovnání je zapsán vždy pouze u prvního meteoru. V tomto režimu bude také vždy poslední výsledek prázdný.

\medskip

Ve druhé části výstupu se nacházejí meteorické roje nalezené ve vstupním souboru, tato část se tedy ve výstupu nachází pouze při spuštění programu v režimu hledání rojů. Každý nalezený meteorický roj zde má sekci nadepsanou klíčovým slovem \texttt{Shower} a kódem prvního meteoru. Na následujících řádcích jsou pak kódy všech meteorů, které do tohoto roje spadají. Způsob vyhodnocení meteorických rojů je popsán v sekci \ref{sec:practical:association}.

\subsection{Příklady příkazů pro spuštění programu\label{sec:practical:examples}}% MARK: Příklady příkazů
Program můžeme spustit zavoláním intepreteru Python \\na soubor \texttt{\_\_main\_\_.py}, který je vstupním bodem programu, nebo na operačních systémech Windows pomocí spustitelného souboru \texttt{meteors.exe}. Souhrnně budeme psát pouze \texttt{meteors}, tedy například
\begin{Verbatim}[commandchars=\\\{\},gobble=4,formatcom=\small]
    {\color{olive}meteors} {\color{gray}--help}
\end{Verbatim}
pro vypsání nápovědy.

\medskip

Mějme vstupní soubor \texttt{meteors.dat} s pozorovanými meteory \\a \texttt{showers.dat} s meteorickými roji.

Náležitost meteorů do těchto rojů můžeme vyhodnotit příkazem
\begin{Verbatim}[commandchars=\\\{\},gobble=4,formatcom=\small]
    {\color{olive}meteors} meteors.dat showers.dat
\end{Verbatim}
Výsledky vyhodnocení budou vypsány do konzole, můžeme je ale nechat uložit do souboru \texttt{results.txt} příkazem
\begin{Verbatim}[commandchars=\\\{\},gobble=4,formatcom=\small]
    {\color{olive}meteors} meteors.dat showers.dat {\color{gray}--output} results.txt
\end{Verbatim}
Alternativně můžeme namísto vlastního souboru s meteorickými roji použít vestavěnou databázi meteorických rojů
\begin{Verbatim}[commandchars=\\\{\},gobble=4,formatcom=\small]
    {\color{olive}meteors} meteors.dat default
\end{Verbatim}

\begin{table}[!h]
    \centering
    \caption{Argumenty programu pro nastavení elementů dráhy meteoru}
    \begin{tabular}[pos]{|l|cccccc|}
        \hline
        \textbf{Element dráhy} & $q$ & $a$ & $e$ & $i$ & $\omega$ & $\Omega$ \\
        \textbf{Argument} & \texttt{-q} & \texttt{-a} & \texttt{-e} & \texttt{-i} & \texttt{-w} & \texttt{-O} \\
        \hline
    \end{tabular}
    \label{tbl:practical:arguments}
\end{table}

Porovnáváme-li pouze jeden meteor, můžeme jej namísto souborem zadat argumenty programu, například
\begin{Verbatim}[commandchars=\\\{\},gobble=4,formatcom=\small]
    {\color{olive}meteors} {\color{gray}-q} 1.0 {\color{gray}-e} 0.9 {\color{gray}-w} 87.6 {\color{gray}-O} 54.3 {\color{gray}-i} 2.1 showers.dat
\end{Verbatim}
I zde zadáváme úhly ve stupních a vzdálenosti v AU. Chceme-li namísto vzdálenosti perihelia $q$ použít délku velké poloosy $a$, nahradíme možnost \texttt{-q} za \texttt{-a}. Jakými argumenty zadat jednotlivé elementy dráhy ukazuje tabulka \ref{tbl:practical:arguments}.

Nakonec, chceme-li pouze nalézt meteorické roje v souboru meteorů, použijeme příkaz
\begin{Verbatim}[commandchars=\\\{\},gobble=4,formatcom=\small]
    {\color{olive}meteors} meteors.dat
\end{Verbatim}

\medskip

Program umí také příkazem
\begin{Verbatim}[commandchars=\\\{\},gobble=4,formatcom=\small]
    {\color{olive}meteors} {\color{gray}--config}
\end{Verbatim}
vygenerovat připravený konfigurační soubour \texttt{.meteorrc}. Ten je použit pouze při spuštění programu ze stejného adresáře, ve kteréme se \texttt{.meteorrc} nachází. Potřebujeme-li konfigurační soubor při spuštění programu ignorovat, můžeme přidat argument \texttt{-0} (to jest nula, nikoliv velké O).

\section{Architektura aplikace}% MARK: Architektura aplikace
Zdrojový kód je sémanticky rozdělen do trojice modulů, každý z nichž se stará o jiný aspekt programu. Vstupním bodem programu je soubour \texttt{\_\_main\_\_.py}, který obstarává pouze základní režii.

\smallskip

Modul \texttt{core} obsahuje, jak název napovídá, jádro aplikace. Tím jsou datové struktury reprezenující orbity, implementace $D$-funkcí na nich a logika porovnávání orbitů. Režijní aktivity zde obstarává submodul \texttt{core.actions}, ve kterém se nachází trojice funkcí odpovídající třem režimům běhů programu. Hlavní režijní aktivitou je zde však řízení vícevláknového porovnávání, které se děje ve funkcích ze submodulu \texttt{core.runners}.

V modulu \texttt{lib} jsou implementovány pomocné nástroje pro načítání souborů a výpis do konzole. Nejdůležitějším je zde submodul \texttt{lib.parser}, který vedle dalších pomocných struktur obsahuje dvě třídy zodpovědné za převod řádků vstupního souboru na programem používané reprezentace orbitů. Jedna z nich je založena na standardní knihovně \texttt{csv}, ta je ovšem momentálně používána pouze pro načítání vestavěné databáze meteorických rojů, která je v od CSV odvozeném formátu TSV. Druhá tato třída byla napsána pro textové soubory s pevnou šířkou sloupců a používá se na uživatelem specifikované vstupní soubory. V submodulu \texttt{lib.io} se nachází, mimo jiné, funkce \texttt{preload()}, která otevírá a kontroluje vstupní soubory a určuje režim běhu programu.

Dále \texttt{meta} je modul pracující s metadaty programu, to jest s konfiguračními soubory a zde i argumenty spuštění programu. Načtení, validaci a zpracování konfigurace obsluhuje submodul \texttt{meta.config}. Zpracovanou konfiguraci z něj poté dostává submodul \texttt{meta.cliargs}, který vyhodnocuje argumenty spušetění programu, kombinuje je s konfigurací a předává zbytku programu instrkuce, co a jak má provést.

\smallskip

Kromě těchto modulů se ve zdrojovém kódu nachází ještě adresář \texttt{constants}, který obsahuje databázi meteorických rojů a schéma konfiguračního souboru. Tyto soubory program načítá dle potřeby a schéma konfiguračního souboru je používáno také chytrými editory.\footnote{Chytré editory používají schéma nikoliv přímo z programu, nýbrž přistupují k souboru v on-line repozitáři se zdrojovým kódem programu.}

\section{Technika zpracování dat}% MARK: Technika zpracování dat
Důležitou součástí návrhu techniky zpracování bylo, aby byl program dostatečně robustní a zvládal vemi velké datové objemy, ale zároveň nebyl zbytečně pomalý -- v algoritmizaci vždy hledáme vhodný kompromis mezi prostorovou a časovou náročností, protože zlepšení jedné obvykle znamená zhoršení druhé.

Vzhledem k prospektu hledání meteorických rojů v potenciálně velmi velkých datových souborech a skutečnosti, že program může zpracovávat data "`off-line,"' tedy ne v reálném čase, byla prioritizována prostorová stabilita: Navrhnout program tak, aby se paměť počítače nezahltila, a to ani u velmi velkých souborů.

\smallskip

Primárním způsobem, jak prostorové stability dosahujeme, je načítání dat ze souborů až ve chvíli, kdy je potřebujeme. Zároveň ale výpočty $D$-funkcí a vyhodnocení kritérií můžeme provádět paralelně, proto porovnávání provádíme na více vláknech, čímž se program zrychluje.

\subsection{Prostorová stabilita}% MARK: Prostorová stabilita
Z hlediska práce s paměťovým prostorem zde máme tři oblasti: srovnávaný soubor, referenční soubor a výstup.

\medskip

Srovnávaný soubor ve všech případech načítáme po blocích. Bloky jsou načítány již z modulu \texttt{core.actions}, tedy na hlavním vlákně, a jsou pro podřízená vlákna společné. Již při načítání každý řádek převádíme na instanci třídy \texttt{core.ast.Orbit}, která obsahuje pouze potřebné údaje, a nikdy tak v paměti neuchováváme více než jeden řádek vstupního souboru.

Pro každý načtený orbit je vytvořena i instance \texttt{core.ast.Result}, která uchovává výsledky porovnání s meteorickými roji. Instance této třídy uchovávají všechny roje, pro které meteor splňuje kritéria, a maximálně pouze tři roje, resp. dokonce jen porovnání, které kritéria nesplňují, ale jsou splnění nejblíže. To, jak blízko splnění kritéria porovnání je, rozhodujeme porovnáváním hodnot
\begin{equation}
    q_{ij}=\frac{D_{ij}}{D_\text{max}}\text{,}
\end{equation}
kde $D_i$ je výsledek $D$-funkce pro $i$-tý meteor a $j$-tý meteorický roj. Čím větší nebo čím blíže jedničce hodnota $q_{ij}$ je, tím blíže je tato dvojice splnění kritéria.

Po dokončení porovnávání bloku se výsledky ukládají na disk. Je-li uživatelem zvolen výstupní soubor, ukládají se přímo do něj, v případě výstupu do konzole je ale přesto vytvořen dočasný soubor pro ukládání výsledků. V průběhu porovnávání však musíme instance \texttt{core.ast.Result} uchovávat v paměti, čímž narážíme na první ze dvou identifikovaných kritických bodů prostorové stability.

\smallskip

Ve výchozím nastavení načítáme do paměti bloky 100 meteorů, které porovnáváme s meteorickými roji v referenčním souboru. Po celou dobu tohoto porovnávání musíme mít v paměti také celý blok výsledků, který nabývá na velikosti s každým dalším meteorickým rojem, tyto výsledky však nemůžeme přímo odkládat na disk, protože porovnáváme několik meteorů najednou a vyžadujeme tedy náhodný zápis do paměti, což soubory neumožňují. A jelikož refernční soubor není teoreticky ve své velikosti nijak omezen, v extrémních případech by mohlo dojít k přetečení paměti na výsledcích porovnání.

Prakticky se ale počet porovnání jednoho meteoru pohybuje v tisících (vestavěná databáze meteorických rojů obsahuje \textasciitilde1000 záznamů a využíváme čtyři $D$-kritéria), což není pro ani slabší počítače v žádném případě problém. Pokud by ale z nějakého důvodu tento bod problematický byl, lze snížit velikost načítaných bloků, což sníží i počet výsledků uchovávaných v paměti.

\medskip

Referenční soubor načítáme po jednotlivých řádcích. To je dáno principem paralelního zpracování, který popisjeme v sekci \ref{sec:practical:parallel}.

\medskip

Pro identifikaci meteorických rojů v souboru meteorů nejprve provádíme porovnání každé z dvojic. Ačkoliv je zde "`srovnávaným"' i "`referenčním"' souborem stejný soubor, stále se držíme principu načítání srovnávaného souboru po blocích a referenčního souboru po řádcích. Hledání meteorických rojů pak provádíme až z výsledků uložených na disku (detailně viz sekce \ref{sec:practical:association}), ovšem pro urychlení tohoto procesu si při ukládání výsledků uchováváme v paměti také polohu ve výstupním souboru, kde se výsledky pro každý meteor nacházejí. Toto považujeme za druhý z kritických bodů paměťové stability programu.

\smallskip

Tyto polohy jsou uchovávány v datové struktuře slovníku, kde jako klíče slouží názvy či kódy meteorů (řetězce) a hodnotami jsou polohy v souboru (celá čísla). Datový objem každého záznamu je tedy minimální, roste ovšem lineárně s velikostí srovnávaného souboru. Pro velmi velké soubory, které by se v případě hledání nových meteorických rojů daly očekávat, můžeme narazit na problémy s pamětí.

Bohužel tento problém nemá řešení, které by drasticky nezpomalilo celý proces. Alternativou je totiž zapisování těchto párů "`klíč--hodnta"' do pomocného souboru na disku, který bychom pak museli pro každý hledaný meteor řádek pro řádku procházet. To je oproti hledání ve slovníku řádově pomalejší a na naprogramování mnohem náročnější.

Přestože hranice, kde se toto stává problematickým, je opravdu velmi vysoko, vnímáme toto jako otevřený problém a jen testování v praxi ukáže, zda je potřeba jej řešit.

\subsection{Paralelní zpracování\label{sec:practical:parallel}}% MARK: Paralelní zpracování
Paralelní zpracování jsme založili na principu, že každé podřízené vlákno pracuje s jedním referenčním orbitem (meteorickým rojem). Každé vlákno si tedy ze souboru načte jeden orbit, se kterým porovná celý blok ze srovnávaného souboru (ten se načítá v hlavním vlákně), poté načte další volný orbit ze souboru a porovná s ním blok a toto opakuje, dokud nenarazí na konec referenčního souboru.

Pro všechna vlákna používáme jedinou instanci \texttt{lib.parser.Parser}, která načítá orbity ze souboru. Ta pomocí programových zámků zajišťuje, že se vlákna v načítání z referenčního souboru střídají, tedy že každé vlákno dostane jiný referenční orbit. Vlákna pak se svým referenčním orbitem porovnávají blok srovnávaných orbitů pomocí všech nastavených kritérií. A jelikož $D$-funkce obsahují nemalé množství goniometrických funkcí, které jsou na výpočet obecně pomalejší, je zde paralelizace výhodná.

Ve výchozím nastavení běží program s osmi podřízenými vlákny.

\medskip

Pro paralelní zpracování musí být připravena také třída \texttt{core.ast.Result}, která zaznamenává výsledky porovnání. Programovými zámky zde zajišťujeme, že při přidávání výsledků porovnání si vlákna vzájemně "`nesahají do paměti."' Toto je důležité obzvláště u neúspěšných porovnání, tedy zamítnutých příslušností do roje, kde se uchovávají pouze tři nejlepší výsledky porovnání a $q$-hodnotu nového výsledku je tedy potřeba porovnat s předchozími uchovávanými zamítnutými roji.

\subsection{Hledání nových meteorických rojů\label{sec:practical:association}}% MARK: Hledání nových meteorických rojů
Hledání nových meteorických rojů je zde založeno na velmi prostém principu zvaném "`serial association"' a je implementováno \\ve funkci \texttt{\_actual\_run\_serial\_assoc()} ze submodulu \texttt{core.runners}. Prakticky hledání provádíme jako hledání do šířky, tedy pomocí fronty. Začínáme s výstupním souborem, který obsahuje výsledky všech porovnání, a slovníkem, který obsahue pozici výsledků každého meteoru v tomto souboru.

Z klíčů slovníku zvolíme první meteor a vytvoříme z něj roj obsahující pouze tento jeden meteor. Ze slovníku pomocí klíče tohoto meteoru získáme jeho polohu ve výstupním souboru a ze slovníku meteor odstraníme. Na dané poloze ze souboru vyčteme meteory, se kterými bylo porovnání úspěšné, a přidáme je do fronty i do meteorického roje. Z fronty následně vyjmeme první položku a opakujeme proces získání polohy ze slovníku, odstranění ze slovníku a načtení dalších meteorů do fronty i do roje.

Celý tento proces se opakuje, dokud se fronta nevyprázdní, tedy dokud nedojdou meteory, které by do tohoto roje náležely. V tomto momentě můžeme roj uložit do souboru s výsledky. Poté, jelikož všechny do tohoto roje náležící meteory jsme ze slovníku odebrali, můžeme opět vybrat první klíč ze slovníku a proces s frontou opakovat. Toto opakujeme, dokud jsou ve slovníku další položky.

\smallskip

Některé takto získané roje mohou obsahovat pouze jeden meteor, v takovém případě se tedy zjevně o žádný roj nejedná. Takovéto "`roje"' je vhodné ani neukládat do souboru s výsledky, kontrolujeme proto počet meteorů v roji ještě před uložením. Můžeme si také určit minimální počet meteorů v roji a zamítat tak "`roje"' obsahující např. jen dva nebo tři meteory.

\medskip

Jsme si vědomi, že tato metoda je značně nedokonalá a bude vesele do jednoho meteorického roje zahrnovat meteory, které mohou pocházet z vícera různých meteorických rojů. Tuto metodu by bylo možné rozšířit například o kontrolu, zda každý z meteorů v roji splňuje $D$-kritérium s jakýmsi středním orbitem tohoto roje, jak navrhovali již \citeauthor{dsh} \cite{dsh}. Ztratili bychom tím ale možnost používat více $D$-kritérií najednou. Jak moc je takováto hlubší kontrola potřeba však může ukázat pouze rozsáhlejší testování na reálných datech.