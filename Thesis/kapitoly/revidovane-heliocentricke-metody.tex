\chapter{Revidované heliocentrické metody}% MARK: Revidované heliocentrické metody

\section{Revidovaná funkce odlišnosti $D_\text{D}$}% MARK: Revidovaná funkce odlišnosti DD
% \cite{remarks} \cite{newapproach} \cite{galligan} \cite{cometassoc}
\citeauthor{cometassoc} při své práci na přiřazování meteorických rojů ke kometám \cite{cometassoc} vytvořil nové $D$-kritérium. Komety jsou jedním z hlavních zdrojů meteorických rojů; jejich úlomky mají podobné dráhy a při střetu se Zemí se objevují ve stejné roční doby ze stejných míst na obloze. Jelikož oběžné dráhy komet jsou dobře změřeny, je možné využít $D$-kritérium také k přiřazení meteorických rojů kometám \cite{cometassoc}. \citeauthor{cometassoc}ova míra orbitální odlišnosti má ovšem jen málo společného s drahami komet, jedná se spíše o vylepšení matematických vlastností $D_{SH}$.

\citeauthor{cometassoc}ovým cílem bylo upravit míru $D_{SH}$ tak, aby všechny sčítance byly bezrozměrné a normalizované. Došel k funkci \cite{cometassoc}\cite{remarks}
\begin{equation}
    \begin{aligned}
        D_D^2(A,B)=\;&\left( \frac{q_B-q_A}{q_A+q_B} \right)^2\\
        &+\left( \frac{e_B-e_A}{e_A+e_B} \right)^2\\
        &+\left( \frac{I^\prime_{AB}}{180^\circ} \right)^2\\
        &+\left( \frac{e_A+e_B}{2} \right)^2\left( \frac{\Theta_{AB}}{180^\circ} \right)^2 \text{,}
    \end{aligned}
\end{equation}


\note{Funkce $D_\text{D}$. Oběžné dráhy komet. Další podle toho, co vše popisuje \textit{Drummond 1981}, až najdu.}

\section{Hybridní funkce $D_\text{H}$}% MARK: Hybridní funkce DH
% \cite{remarks}
\note{Analýza matematického chování $D_\text{SH}$ a $D_\text{D}$ dle \cite{remarks}. Sloučení obou přístupů do funkce $D_\text{H}$.}