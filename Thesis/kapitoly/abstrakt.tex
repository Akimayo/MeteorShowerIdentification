{\small
Z fotografických, video či radarových pozorování meteorů jsme analýzou trajektorie meteoru schopni určit orbit meteoroidu před jeho setkáním se Zemí. Na základě elementů dráhy tohoto orbitu určujeme, do kterého meteorického roje daný meteor náleží.\\
Za tímto účelem bylo v druhé polovině minulého století navrženo několik metod, hromadně označované jako $D$-kritéria, které vycházejí z metody představené Southworthem a Hawkinsem. Ta spočívá ve výpoču míry orbitální odlišnosti, to jest číselné hodnoty, která reprezentuje, jak moc se od sebe orbity dvou meteoroidů liší, a jejím následným porovnáním s jistou pevnou hraniční hodnotou. Pokud je míra menší než hraniční hodnota, náleží dané meteory do stejného meteorického roje.\\
V této práci popisujeme kritéria $D_\text{SH}$, $D_\text{D}$, $D_\text{H}$ a $D_\text{N}$ a diskutujeme jejich vlastnosti. V rámci praktické části jsme v jazyce Python vytvořili vlastní programový nástroj, který tato kritéria aplikuje na reálná data. Detailně se také zabýváme technikou pořízení a zpracování fotografických a video záznamů meteorů.
}