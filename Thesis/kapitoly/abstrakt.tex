Z pozorování meteorů jsme analýzou trajektorie meteoru schopni určit dráhu meteoroidu. Na základě elementů této dráhy určujeme, do kterého meteorického roje meteor náleží. Za tímto účelem byly navrženy metody hromadně označované jako $D$-kritéria, které spočívají ve výpočtu míry orbitální odlišnosti a jejím následným porovnáním s jistou pevnou hraniční hodnotou. Kritéria $D_\text{SH}$, $D_\text{D}$, $D_\text{H}$ a $D_\text{N}$ popisujeme a diskutujeme jejich vlastnosti. Poznatky využíváme k tvorbě vlastního programového nástroje, který tato kritéria aplikuje na reálná data.