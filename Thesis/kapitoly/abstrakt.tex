Z pozorování meteorů jsme analýzou trajektorie meteoru schopni určit orbit meteoroidu. Na základě elementů dráhy orbitu určujeme, do kterého meteorického roje meteor náleží. Za tímto účelem byly navrženy metody hromadně označované jako $D$-kritéria, která spočívají ve výpoču míry orbitální odlišnosti a jejím následným porovnáním s jistou pevnou hraniční hodnotou. Kritéria $D_\text{SH}$, $D_\text{D}$, $D_\text{H}$ a $D_\text{N}$ popisujeme a diskutujeme jejich vlastnosti. Poznatky využíváme k tvorbě vlastního programového nástroje, který tato kritéria aplikuje na reálná data.