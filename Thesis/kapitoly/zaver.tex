\chapwithtoc{Závěr}% MARK: Závěr
V této práci jsme se seznámili s metodami primárně fotografického a video pozorování meteorů a u fotografických měření jsme důkladně prozkoumali techniku určení polohy a rychlosti meteoru z pozorování ze dvou stanic. Znalost polohy na obloze a rychlosti nám umožnila spočítat elementy dráhy meteoroidu před vstupem do zemské atmosféry.

\medskip

Elementy dráhy jsou rozhodujícím faktorem pro přiřazování meteorů do meteorických rojů. Ke zjištění, zda meteor náleží do daného meteorického roje, využíváme $D$-kritéria, jejichž princip navrhli \citeauthor{dsh}. Popsali jsme zde čtveřici těchto kritérií; od původního $D_\text{SH}$ přes modifikovaná $D_\text{D}$ a $D_\text{H}$ až po nejnovější $D_\text{N}$, které na rozdíl od předchozích kritérií pracuje s geocentrickými, nikoliv heliocentrickými, veličinami.

Na základě \citeauthor{galligan}ových simulací a vlastního testování jsme porovnali úspěšnost jednotlivých kritérií: Ze simulací vychází jako nejúspěšnější a matematicky nejlepší kritérium $D_\text{N}$, to se však při našem testování na reálných datech ukázalo jako příliš "`svolné"' v přiřazování meteorů do meteorických rojů. Unikátní přístup tohoto kritéria je ale slibný hlavně z hlediska časové stability v řádech desítek tisíc let, kde ostatní kritéria již dávno selhávají. Ukázali jsme také, že dvojice nejstarších kritérií, $D_\text{SH}$ a $D_\text{D}$, jsou naprosto obstojné, z nich odvozené $D_\text{H}$ ale bohužel i přes své vylepšené teoretické vlastnosti v praktické použitelnosti pokulhává.

\medskip

V poslední kapitole jsme povrchově i detailně popsali programový nástroj, který jsme v rámci této práce vytvořily. Jeho úkolem je zde popsaná $D$-kritéria používat k přiřazování meteorů z reálných pozorování do ustanovených meteorických rojů, či případně v reálných datech hledat nové meteorické roje.

Jedná se o program napsaný v jazyce Python, měl by tedy být své cílové skupině uživatelů, to jest fyzikům, srozumitelný. Byl navržen tak, aby co nejlépe zvládal i velké vstupní soubory s pozorováními, které se obzláště pro hledání nových meteorických rojů dají očekávat. Jak v tomto uspěje se však ukáže až při reálném používáni, jelikož byl testován pouze na relativně malém vzorku dat, který nám byl pro jeho vývoj poskytnut.

\smallskip

Pro nás se také jednalo o první projekt v jazyce Python. I přes úvodní souboje s některými jeho vlastnostmi, jeho způsoby implementace některých v programování používaných triků a jeho silnou nevolí vůči robustní architektuře aplikace se nám ale, doufáme, podařilo vytvořit funkční, dobře vypadající a uživatelsky přívětivý program s přehledným a dobře dokumentovaným zdrojovým kódem.
% 
% Program včetně zdrojového kódu je přiložen v elektronických přílohách této práce a veřejně dostupný v repozitáři \href{https://github.com/Akimayo/MeteorShowerIdentification}{github.com/Akimayo/MeteorShowerIdentification}.