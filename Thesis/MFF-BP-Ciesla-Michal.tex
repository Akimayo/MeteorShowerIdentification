\documentclass[]{./MFFPrace}

\usepackage[starfontserif]{starfont}
\usepackage{caption}

\NazevPrace{Identifikace meteorických rojů}
\NazevPraceEN{Meteor Shower Identification}
\AutorPrace{Michal~Ciesla}
\RokOdevzdani{2024}
\Katedra{Astronomický ústav AV ČR}
\KatedraEN{Astronomical~Institute, Academy of Sciences of the Czech Republic}
\TypPracoviste{Ústav}
\TypPracovisteEN{Institute}
\Vedouci{RNDr.~Pavel~Koten,~Ph.D.}
\KatedraVedouciho{\IKatedra}
\KatedraVedoucihoEN{\IKatedraEN}
\StudijniProgram{Fyzika}
\StudijniObor{FP\ask{Opravdu jen takto zkratkou?}}
\Podekovani{\todo{Napsat poděkování}}
\Abstrakt{Z pozorování meteorů jsme analýzou trajektorie meteoru schopni určit orbit meteoroidu. Na základě elementů dráhy orbitu určujeme, do kterého meteorického roje meteor náleží. Za tímto účelem byly navrženy metody hromadně označované jako $D$-kritéria, která spočívají ve výpoču míry orbitální odlišnosti a jejím následným porovnáním s jistou pevnou hraniční hodnotou. Kritéria $D_\text{SH}$, $D_\text{D}$, $D_\text{H}$ a $D_\text{N}$ popisujeme a diskutujeme jejich vlastnosti. Poznatky využíváme k tvorbě vlastního programového nástroje, který tato kritéria aplikuje na reálná data.}
\AbstraktEN{{\small
From photographic, video, or radar observations of meteors we are able to determine the orbit of a meteorite before it's encounter with Earth by analysing the trajectory of the meteor. We then decide which meteor shower the observed meteor belongs to based on the orbital elements of this orbit.\\
In the second half of the last century, several methods, collectively known as $D$-criteria, have been devised for this exact purpose. These have stemmed from the method presented by Southworth and Hawkins, which calculates an orbital dissimilarity measure, a numerical value representing how different orbits of two meteoroids are, and compares it with some fixed cutoff value. If the measure is less than the cutoff, the meteors belong to the same meteor shower.\\
In this thesis we describe the $D_\text{SH}$, $D_\text{D}$, $D_\text{H}$ and $D_\text{N}$ criteria and discuss their properties. As the practical part, we have created a software tool written in Python which is used to apply these criteria to real-world data. We also explain the techniques of capturing and analysis of photographic and video records of meteors in great detail.
}}
\KlicovaSlova{{meteorické roje}, {$D$-kritérium}, {míra orbitální odlišnosti} \todo{Vypsat klíčová slova}}
\KlicovaSlovaEN{{meteor swarms}, {$D$-criterion}, {orbital dissimilarity measure} \todo{Write out keywords}}

\addbibresource{literatura.bib}

\captionsetup{justification=centering}

\begin{document}
    \maketitle
    \tableofcontents
    \begin{todolist}% MARK: To-Do list
        \pagebreak
        \item[\done] Úvod
        \item[\done] Observace, měření a reprezentace
        \begin{todolist}
            \item[\done] Elementy dráhy
            \item[\done] Fotografická měření
        \end{todolist}
        \item Historické metody klasifikace
        \begin{todolist}
            \item[\done] Přímé porovnávání elementů dráhy
            \item[\done] Funkce orbitální odlišnosti, $D$-kritérium
        \end{todolist}
        \item Revidované heliocentrické metody
        \begin{todolist}
            \item Revidovaná funkce odlišnosti $D_\text{D}$ a příslušnost kometám
            \item Hybridní funkce $D_\text{H}$
        \end{todolist}
        \item Geocentrický přístup
        \begin{todolist}
            \item Semi-invariantní orbitální veličiny
            \item Geocentrická funkce $D_\text{N}$
        \end{todolist}
        \item Porovnání úspěšnosti metod klasifikace
        \item Praktická část
        \item Závěr
    \end{todolist}
    \noindent
    
    \chapwithtoc{Úvod}
\note{Ujasnit základní terminologii, obzvláště to, že meteory z definice protínají oběžnou dráhu Země. Potenciálně uvést souvislost meteorických rojů s kometami, na to ale bude vlastní sekce. Konkretizovat rozsah práce na matematicko-fyzikální metody, nebude se zabývat machine-learning přístupy -- ty je téměř nemožné analyzovat.}

    \chapter{Observace, měření a reprezentace}
K pozorování meteorů a jejich záznamu se v současnosti používají tři přístupy: fotografické snímky, videozáznamy a radarová měření.

Ve všech třech přístupech je cílem sledovat a zaznamenávat celou oblohu nebo alespoň její velkou část. Například u fotografického přístupu se používá rybí oko \cite{ceplecha} -- čočka nebo objektiv, který je schopný zobrazit celou oblohu na jeden snímek. Cenou za takto širokoúhlý snímek je velké zkreslení obrazu, to lze ale pro účely měření matematicky odstranit \cite{ceplecha}.

\todo{Radarová měření dle \cite{radiosurvey}}

Fotografické snímky a videozáznamy jsou z hlediska měření velmi blízké: Videozáznam je v principu pouze série fotografií, v minulém století se ale využívalo spíše opačného přístupu, kdy se několik fotografických záběrů zaznamenalo na jeden snímek \cite{ceplecha}. Oba přístupy tedy dávají průběh polohy (a případně i luminosity) meteoru v čase. Konkrétně pro metody identifikace meteorických rojů potřebujeme právě dráhu (polohu) a rychlost meteoru \cite{ceplecha}, abychom zjistili orbitální dráhu meteoroidu. Důkladněji se zpracování fotografických snímků budeme věnovat v sekci \ref{sec:foto}.

\section{Elementy dráhy}
\note{Rozebrat geometrii orbitálních drah a přehledně rozepsat a rozkreslit význam elementů dráhy.}
Orbitální dráhy jsou v prvním přiblížení elipsy, které jsou nakloněné v prostoru. Jedním z ohniskových bodů je vždy těžiště (Sluneční) soustavy, pro určení dráhy tedy stačí 5 parametrů \cite{newapproach}.

Dva z parametrů popisují tvar elispy; její velikost a excentricitu \cite{newapproach}. Jako excentricitu používáme lineární excentricitu $e$. Excentricita náleží do intervalu $e\in\left[0,1\right]$,\footnote{Excentricita může být také $>1$ pro hyperbolické dráhy. \ask{Hyperbolické, předpokládám, ignorujeme, protože by patřily do sporadického pozadí automaticky (nemají mateřský objekt ve Sluneční soustavě).}} a udává, jak blízká kružnici tato dráha je (viz obrázek \ref{img:excentricity}).

\todo{Ilustrace excentricit}

Druhý parametr udává velikost, poloměr, elipsy. Zde používáme buďto délku hlavní poloosy $a$ nebo vzdálenost perihelionu $q$ (efektivně vzdálenost okraje elipsy od ohniska). Jejich vztah ilustruje obrázek \ref{img:elipsa} a mezi oběma lze převádět pomocí rovnice \cite{ceplecha}
$$
    q=a(1-e)\text{.}
$$

\todo{Obrázek vztahu mezi $a$ a $q$}

\section{Fotografická měření\label{sec:foto}}
\note{}
    \chapter{Historické metody klasifikace}

\section{Přímé porovnávání elementů dráhy}
% \cite{radiosurvey}
\note{Metoda klasifikace z \cite{radiosurvey}.}

\section{Funkce orbitální odlišnosti, $D$-kritérium}
% \cite{dsh} \cite[220]{radiosurvey} \cite[604]{remarks} \cite{newapproach} \cite[623]{galligan}
\note{\cite{dsh}. Obecná míra odlišnosti. Míra odlišnosti elementů dráhy. Odlišnost dvou orbitů a odlišnost od průměrného orbitu.}
    \chapter{Revidované heliocentrické metody}% MARK: Revidované heliocentrické metody

\section{Revidovaná funkce odlišnosti $D_\text{D}$}% MARK: Revidovaná funkce odlišnosti DD
\citeauthor{cometassoc} při své práci na přiřazování meteorických rojů ke kometám \cite{cometassoc} vytvořil nové $D$-kritérium. Komety jsou jedním z hlavních zdrojů meteorických rojů; jejich úlomky mají podobné dráhy a při střetu se Zemí se objevují ve stejné roční doby ze stejných míst na obloze. Jelikož oběžné dráhy komet jsou dobře změřeny, je možné využít $D$-kritérium také k přiřazení meteorických rojů kometám \cite{cometassoc}. \citeauthor{cometassoc}ova míra orbitální odlišnosti má ovšem jen málo společného s drahami komet, jedná se spíše o vylepšení matematických vlastností $D_\text{SH}$.

\citeauthor{cometassoc}ovým cílem bylo upravit míru $D_\text{SH}$ tak, aby všechny sčítance byly bezrozměrné a normalizované. Došel k funkci \cite{cometassoc}\cite{remarks}
\begin{equation}
    \begin{aligned}
        D_\text{D}^2(A,B)=\; & \left( \frac{q_B-q_A}{q_A+q_B} \right)^2                                                   \\
                             & +\left( \frac{e_B-e_A}{e_A+e_B} \right)^2                                                  \\
                             & +\left( \frac{I^\prime_{AB}}{180^\circ} \right)^2                                          \\
                             & +\left( \frac{e_A+e_B}{2} \right)^2\left( \frac{\Theta_{AB}}{180^\circ} \right)^2 \text{,}
    \end{aligned}
    \label{eqn:revised:d_d}
\end{equation}
kde úhel $I^\prime_{AB}$ mezi rovinami orbitů, který nahrazuje délku tětivy $I_{AB}$ z funkce $D_\text{SH}$, je získán výpočtem \cite{cometassoc}
\begin{equation}
    \cos{I_{AB}}=\cos{i_A}\cos{i_B}+\sin{i_A}\sin{i_B}\cos{\Omega_B-\Omega_A}
\end{equation}
a úhel $\Theta_{AB}$ je skutečný úhel mezi spojnicemi apsid\footnote{Spojnice apsid spojuje perihelium a aphelium; zjednodušeně a prakticky ekvivalentně si ji můžeme představit jako přímku procházející periheliem a Sluncem.} vypočtený z jejich ekliptikálních souřadnic $\beta^\prime,\lambda^\prime$ vzorcem \cite{cometassoc}
\begin{equation}
    \cos{\Theta_{AB}}=\sin{\beta^\prime_A}\sin{\beta^\prime_B}+\cos{\beta^\prime_A}\cos{\beta^\prime_B}\cos{\lambda^\prime_B-\lambda^\prime_A} \text{.}
\end{equation}

Ekliptikální délka a šířka perihelia se vypočte z elementů dráhy podle vzorců \cite{cometassoc}
\begin{eqnarray}
    \beta^\prime &=& \arcsin{\left( \sin{i}\sin{\omega} \right)}\\
    \lambda^\prime &=& \Omega + \arctan{\left( \cos{i}\tan{\omega} \right)} \text{.}
\end{eqnarray}
K $\lambda^\prime$ ještě přičteme $180^\circ$, pokud $\cos{\omega}>0$ \cite{cometassoc}.

\smallskip

Analyticky můžeme zjistit, že $D_\text{D}$ nabývá hodnot z intervalu $\left[ 0;\sqrt{3{,}25} \right]$ \cite{cometassoc}. \citeauthor{cometassoc} používá pro kritérium příslušnosti hranici $D_\text{max}=0{,}105$ \cite{cometassoc} a tato hranice se i pozdějšími testy ukázala býti vhodnou; \cite{galligan} doporučuje maximální hodnoty
$$
    D_\text{D} \le \begin{cases}
        0{,}09 & i < 10^\circ             \\
        0{,}11 & 10^\circ\le i < 90^\circ \\
        0{,}18 & i \ge 90^\circ \text{.}  % Retrográdní orbity
    \end{cases}
$$

\section{Hybridní funkce $D_\text{H}$}% MARK: Hybridní funkce DH
Numerickým chováním funkcí $D_\text{SH}$ a $D_\text{D}$ se zabýval \citeauthor{remarks}. Zvolil referenční orbit, jehož několik tisíc kopií simulovaně perturboval náhodnými impulsy síly podobným způsobem, jako se tomu děje ve Sluneční soustavě. Studoval pak závislosti jednotlivých členů obou funkcí v závislosti na excentricitě \cite{remarks}, jak ukazuje obrázek \ref{img:revised:jopek}, kde jsou jednotlivé perturbované kopie rozděleny do binů dle excentricity a zprůměrovány \cite{remarks}.

\begin{figure}[ht]
    \centering
    \subfigure[vzdálenosti perihelia $q$]{\includegraphics[width=0.4\linewidth]{img/plots/remarks-q.png}\label{img:revised:jopek:1}}\hfill
    \subfigure[excentricity $e$]{\includegraphics[width=0.4\linewidth]{img/plots/remarks-e.png}\label{img:revised:jopek:2}}
    \caption[Závislost členů $D$-kritérií na excentricitě]{
        Závislost členů $D$-kritérií na excentricitě\\
        {\small (zdroj: \cite{remarks})}
    }
    \label{img:revised:jopek}
\end{figure}

Cílem je, aby se příspěvek jednotlivých členů choval co nejvíce konstantě vzheledem k excentricitě. První, graf \ref{img:revised:jopek:1}, ukazuje, že $D_\text{SH}$ se příliš liší pro různé hodnoty $q$ a pro excentricity blízké jedné utíká do příliš vysokých hodnot. $D_\text{D}$ je v tomto ohledu stabilnější, tedy přidání váhy $1/(q_A+q_B)$ členu $(q_B-q_A)$ je zde přínosné. Naopak graf \ref{img:revised:jopek:2} ukazuje, že ve členu excentricit je mnohem stabilnější prostý rozdíl $(e_B-e_A)$ s váhou $1$ z funkce $D_\text{SH}$.

\medskip

Výsledkem této práce byla nová míra orbitální odlišnosti $D_\text{H}$, která je hybridem $D_\text{SH}$ a $D_\text{D}$ odstraňujícím vysokou citlivost obou měr na excentricitu \cite{remarks}. Tato míra má předpis \cite{remarks}
\begin{equation}
    \begin{aligned}
        D_\text{H}^2= & \left( \frac{q_B-q_A}{q_A+q_B} \right)^2                                    \\
                      & +\left( e_B-e_A \right)^2                                                   \\
                      & +\left( 2\sin{\frac{I_{AB}}{2}} \right)^2                                   \\
                      & +\left( \frac{e_A+e_B}{2} \right)^2\left( 2\sin{\frac{\Pi_{AB}}{2}} \right) \text{,}
    \end{aligned}
    \label{eqn:revised:d_h}
\end{equation}
kde $I_{AB}$ spočteme vzorcem \eqref{eqn:history:i_ba} a $\Pi_{AB}$ vzorcem \eqref{eqn:history:pi_ba}. Vhodnými hranicemi pro toto kritérium jsou \cite{galligan}
$$
    D_\text{H} \le \begin{cases}
        0{,}10 & i < 10^\circ \\
        0{,}16 & 10^\circ \le i < 90^\circ \text{.}
    \end{cases}
$$
    \chapter{Geocentrický přístup}% MARK: Geocentrický přístup

\section{Semi-invariantní orbitální veličiny}% MARK: Semi-invariantní orbitální veličiny
% \cite{newapproach}
\note{Zavedení sekulárních semi-invariantů. Rozbor invariance dle \cite{newapproach}. Potenciálně detailnější popis semi-invariantů dle dalších zdrojů.}

\section{Geocentrická funkce $D_\text{N}$}% MARK: Geocentrická funkce DN
% \cite{newapproach}
\note{Funkce odlišnosti $D_\text{N}$ zavedená na sekulárních semi-invariantech dle \cite{newapproach}.}
    \chapter{Porovnání úspěšnosti metod klasifikace}% MARK: Porovnání úspěšnosti metod klasifikace
% \cite{galligan}
\note{Shrnutí analýzy $D_\text{SH}$ a $D_\text{D}$ z \cite{remarks}. Experimentální(?) poznatky z \cite{galligan} a soupis vhodných rozsahů pro jednotlivé metody. Výhody $D_\text{N}$.}
    \chapter{Programový nástroj pro identifikaci rojů}% MARK: Praktická část

    \chapwithtoc{Závěr}% MARK: Závěr
V této práci jsme se seznámili s metodami primárně fotografického a video pozorování meteorů. U fotografických měření jsme důkladně prozkoumali techniku určení polohy a rychlosti meteoru z pozorování ze dvou stanic. Znalost polohy na obloze a rychlosti nám umožnila spočítat elementy dráhy meteoroidu před vstupem do zemské atmosféry.

\medskip

Elementy dráhy jsou rozhodujícím faktorem pro přiřazování meteorů do meteorických rojů. Ke zjištění, zda meteor náleží do daného meteorického roje, využíváme $D$-kritéria, jejichž princip navrhli \citeauthor{dsh}. Popsali jsme zde čtveřici těchto kritérií; od původního $D_\text{SH}$ přes modifikovaná $D_\text{D}$ a $D_\text{H}$ až po nejnovější $D_\text{N}$, které na rozdíl od předchozích kritérií pracuje s geocentrickými, nikoliv heliocentrickými, veličinami.

Na základě \citeauthor{galligan}ových simulací a vlastního testování jsme porovnali úspěšnost jednotlivých kritérií: Ze simulací vychází jako nejúspěšnější a matematicky nejlepší kritérium $D_\text{N}$, to se však při našem testování na reálných datech ukázalo jako příliš "`svolné"' v přiřazování meteorů do meteorických rojů. Unikátní přístup tohoto kritéria je ale slibný hlavně z hlediska časové stability v řádech desítek tisíc let, kde ostatní kritéria již dávno selhávají. Ukázali jsme také, že dvojice nejstarších kritérií, $D_\text{SH}$ a $D_\text{D}$, jsou naprosto obstojné, z nich odvozené $D_\text{H}$ ale bohužel i přes své vylepšené teoretické vlastnosti v praktické použitelnosti pokulhává.

\medskip

V poslední kapitole jsme na přehledové i technické úrovni popsali programový nástroj, který jsme v rámci této práce vytvořili. Jeho úkolem je zde popsaná $D$-kritéria používat k přiřazování meteorů z reálných pozorování do ustanovených meteorických rojů, či případně v reálných datech hledat nové meteorické roje.

Jedná se o program napsaný v jazyce Python, měl by tedy být své cílové skupině uživatelů, to jest fyzikům, srozumitelný. Byl navržen tak, aby co nejlépe zvládal i velké vstupní soubory s pozorováními, které se obzvláště pro hledání nových meteorických rojů dají očekávat. Jak v tomto uspěje se však ukáže až při reálném používáni, jelikož byl testován pouze na relativně malém vzorku dat, který nám byl pro jeho vývoj poskytnut.

\smallskip

Pro nás se také jednalo o první projekt v jazyce Python. I přes úvodní souboje s některými jeho vlastnostmi se nám podařilo vytvořit funkční, dobře vypadající a uživatelsky přívětivý program s přehledným a dobře dokumentovaným zdrojovým kódem. Program včetně zdrojového kódu je přiložen v elektronických přílohách této práce a veřejně dostupné v repositáři \\\href{https://github.com/Akimayo/MeteorShowerIdentification}{https://github.com/Akimayo/MeteorShowerIdentification}.

    \printbibliography[title=Seznam použité literatury]% MARK: Seznam použité literatury
    \ask{Jak co nejvhodněji použít jména vs. iniciály? Jopek a Southworth mají jména dostupná; u Galligana, Nilssona a Hawkinse jsem bohužel ani po rozsáhlejším pátraní v publikacích a na webech univerzit nenašel.}

    \noindent
    \note{\textbf{Statistics of Meteor Streams} \cite{dsh}: míra orbitální odlišnosti $D(A,B)$, odlišnost od průměru $D(M,N)$, zanedbatelnost $\nu$, závislost orbitálních proměnných, odhad náhodných přiřazení orbitů do rojů, klasifikace rojů (major/minor/spurious), tabulky orbitálních parametrů, náležitost rojů ke kometám, perturbace}\\
    \note{\textbf{A Southern Hemisphere Radio Survey...} \cite{radiosurvey}: technika měření, podmínky seskupování orbitů do rojů, vyřazení malých skupin, podobnost ``minor'' a ``major'' rojů, dvojí detekce rojů s nízkou inklinací}\\
    \note{\textbf{Remarks...} \cite{remarks}: příklad fiktivních orbitů protínající oběžnou dráhu Země, úměrnost $D_\text{SH}$ rychlosti, Drummondova funkce odlišnosti, nežádoucí jevy $D_\text{SH}$ a $D_\text{D}$, rozbor oblouků vs. tětiv ve vzorcích, hybridní funkce odlišnosti $D_\text{H}$}\\
    \note{\textbf{...a new approach} \cite{newapproach}: Keplerovské souřadnice, lineární závislost Keplerovských souřadnic v $D_\text{SH}$, geocentrický přístup, semi-invariantní veličiny, geocentrická funkce odlišnosti $D_\text{N}$}\\
    \note{\textbf{Performance of the $D$-criteria...} \cite{galligan}: hranice hodnot odlišnosti pro různé inklinace, sporadické pozadí, porovnání účinnosti metod}\\
    \note{\textbf{...Comet and Meteor Shower Association} \cite{cometassoc}: příslušnost některých meteorických rojů kometám, tabulka elementů dráhy těchto komet, poznámky k $D_\text{SH}$, lepší rozpis $D_\text{D}$}\\
    \note{\textbf{...Photographic Fireball Networks} \cite{ceplecha}: metody fotografického záznamu meteorů, definice radiantu, výpočet orbitu, Gaussova gravitační konstanta $k=0{,}01720209895$}\\
    \note{\textbf{Definitions...in meteor astronomy} \cite{meteorastro}: definice pojmů meteorit, meteoroid a meteor}\\

    \noindent
    \ask{DRUMMOND, J. D. 1979. On the meteor/comet orbital discriminant D. \textit{Proc. Southwest Reg. Conf. Aslron. Astrophys.} \textbf{5, 83-86}.}

    \listoffigures
    \listoftables
\end{document}